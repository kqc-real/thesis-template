% !TEX root = ../Bachelor-Thesis.tex

\chapter{Implementierung und Ergebnisse}
\label{ch:realisierung}

\section{Implementierungsdetails}

In diesem Kapitel werden die praktischen Aspekte der Umsetzung dokumentiert.

\subsection{Technologiestack}

Folgende Technologien wurden für die Implementierung eingesetzt:

\begin{itemize}
  \item Programmiersprache: Python 3.10+
  \item Framework: Django / FastAPI
  \item Datenbank: PostgreSQL
  \item Versionskontrolle: Git
\end{itemize}

\subsection{Architektur der Lösung}

Die Implementierung folgt dem in Kapitel \ref{ch:konzept} beschriebenen Design.

\section{Experimentelles Setup}

Die Validierung erfolgt anhand von realistischen Testszenarien.

\subsection{Testumgebung}

\begin{itemize}
  \item Hardware: Intel i7, 16GB RAM
  \item Betriebssystem: Ubuntu 22.04
  \item Testdaten: Synthethische und reale Datensätze
\end{itemize}

\section{Ergebnisse}

Die durchgeführten Experimente zeigen folgende Ergebnisse:

\subsection{Leistungsmessungen}

Die entwickelte Lösung erreicht eine durchschnittliche Ausführungszeit von \(t_{avg} = 150ms\) mit einer Standardabweichung von \(\sigma = 25ms\).

\subsection{Qualitätsmetriken}

\begin{itemize}
  \item Genauigkeit: 95\%
  \item Verlässlichkeit: 99.2\%
  \item Skalierbarkeit: Linear bis 10.000 Requests/min
\end{itemize}

\section{Vergleich mit existierenden Ansätzen}

Die entwickelte Lösung übertrifft bestehende Ansätze in folgenden Aspekten:

\begin{itemize}
  \item 20\% schneller als Baseline
  \item 15\% geringerer Speicherbedarf
  \item Bessere Fehlertoleranz
\end{itemize}

\section{Validierung und Verifikation}

Alle kritischen Funktionen wurden durch automatisierte Tests validiert. Die Testabdeckung beträgt 92\%.

