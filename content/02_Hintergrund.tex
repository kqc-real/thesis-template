% !TEX root = ../Bachelor-Thesis.tex

\chapter{Theoretischer Hintergrund}
\label{ch:hintergrund}

\section{Grundkonzepte}

Dieses Kapitel vermittelt die notwendigen Grundlagen zum Verständnis der Arbeit. Die behandelten Konzepte bilden die theoretische Basis für die später entwickelten Lösungsansätze \cite{smith2020example}.

\subsection{Konzept 1: Fundamentale Begriffe}

Zunächst werden die zentralen Begriffe erläutert und definiert, die im Kontext dieser Arbeit verwendet werden.

\subsection{Konzept 2: Etablierte Methoden}

Etablierte Methoden und Techniken werden dargestellt, um den aktuellen Stand der Technik zu verdeutlichen \cite{doe2019research}.

\section{Verwandte Arbeiten}

In diesem Abschnitt wird ein Überblick über relevante Forschungsarbeiten und Publikationen gegeben.

\subsection{Ansatz A}

Der erste bekannte Lösungsansatz wird kritisch analysiert mit seinen Vorteilen und Limitierungen.

\subsection{Ansatz B}

Ein alternativer Ansatz wird dargestellt und mit dem vorherigen verglichen \cite{lee2018conference}.

\section{Vergleich und Bewertung}

Die verschiedenen Ansätze werden systematisch verglichen und bewertet. Daraus ergibt sich die Notwendigkeit eines neuen oder verbesserten Ansatzes.

\section{Forschungslücke}

Basierend auf der Analyse können folgende Lücken identifiziert werden:

\begin{itemize}
  \item Limitation 1: Bestehende Ansätze behandeln diesen Aspekt nicht
  \item Limitation 2: Skalierungsprobleme in praktischen Anwendungen
  \item Limitation 3: Mangelnde Validierung in Echtszenarien
\end{itemize}

Diese Arbeit trägt dazu bei, diese Lücken zu schließen.

