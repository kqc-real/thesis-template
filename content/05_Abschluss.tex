% !TEX root = ../Thesis.tex

\chapter{Fazit und Ausblick}
\label{ch:abschluss}

\section{Zusammenfassung der Ergebnisse}

Diese Arbeit demonstriert, wie eine agentische Architektur für Software Engineering gestaltet, implementiert und evaluiert werden kann.\footnote{Vergleiche dazu \cite{lee2018conference} und \cite{website2021example} für aktuelle Forschungstrends.} Die Ergebnisse stützen die in Kapitel \ref{ch:konzept} vorgestellten Prinzipien (Planung, Tool-Nutzung, Gedächtnis, Safety) und die in Kapitel \ref{ch:realisierung} gezeigte Realisierung.

\subsection{Beantwortung der Forschungsfragen}

\textbf{Forschungsfrage 1:} Wie können existierende Methoden verbessert werden?

Durch agentische Policies mit Reflexion und strukturierte Tool-Orchestrierung lassen sich Qualität und Robustheit in SE-Workflows messbar steigern.

\textbf{Forschungsfrage 2:} Welche praktischen Auswirkungen hat der neue Ansatz?

In realistisch simulierten Szenarien (Tests, Linting, Refactoring) zeigen sich Effizienzgewinne bei gleichzeitig verbesserter Nachvollziehbarkeit.

\section{Beiträge dieser Arbeit}

Diese Arbeit leistet folgende Beiträge zur Spezialisierung \enquote{Software Engineering mit agentic AI}:

\begin{enumerate}
  \item Referenzarchitektur für agentische SE-Workflows (Planung, Tools, Gedächtnis, Safety)
  \item Praxisnahe Implementierung inkl. Beispiel-Listing und Evaluationskriterien
  \item Ableitung von Leitlinien für Testbarkeit, Sicherheit und Kostenkontrolle
  \item Übertragbarkeit auf ähnliche SE-Szenarien (Code-Review, Testgenerierung)
\end{enumerate}

\section{Limitierungen}

Trotz der positiven Ergebnisse gibt es folgende Limitierungen:

\textquote[\cite{lee2018conference}]{Während agentische Systeme vielversprechend sind, müssen ihre Grenzen in kontrollierten Umgebungen sorgfältig evaluiert werden, bevor sie in Produktionssystemen eingesetzt werden.}

Kritisch anzumerken ist auch die gesellschaftliche Dimension:

\begin{quote}
\textit{Die Automatisierung von Software-Engineering-Aufgaben birgt erhebliche Risiken für den Arbeitsmarkt. Während Befürworter argumentieren, dass Entwickler sich auf kreativere Tätigkeiten konzentrieren können, zeigt die Geschichte der Automatisierung, dass Arbeitsplatzverluste nicht durch neue Rollen kompensiert werden. Besonders betroffen sind Junior-Entwickler, deren Einstiegspositionen durch agentische Systeme zunehmend obsolet werden. Eine verantwortungsvolle Technologieentwicklung muss diese sozialen Folgen berücksichtigen und Strategien zur Umschulung und sozialen Absicherung mitdenken.}
\smallskip
\begin{flushright}
— Eve Miller, \textit{Handbook of Examples} \cite{miller2017chapter}
\end{flushright}
\end{quote}

\begin{itemize}
  \item Die Experimente wurden in kontrollierter Umgebung durchgeführt
  \item Skalierungstests waren auf 10.000 Requests/min begrenzt
  \item Die Validierung konzentrierte sich auf spezifische Datensätze
\end{itemize}

\section{Zukünftige Arbeiten}

Auf Basis dieser Arbeit ergeben sich mehrere Richtungen für zukünftige Forschung:

\subsection{Kurzfristige Verbesserungen}

\begin{itemize}
  \item Optimierung der Speichernutzung für Echtzeit-Anwendungen
  \item Erweiterung der Testabdeckung auf weitere Datensätze
  \item Integration mit bestehenden Systemen
\end{itemize}

\subsection{Langfristige Perspektiven}

\begin{itemize}
  \item Erweiterung des Ansatzes auf verwandte Problemdomänen
  \item Untersuchung von Hybrid-Methoden
  \item Machine-Learning basierte Optimierungen
\end{itemize}

\section{Schlusswort}

Diese Arbeit trägt zu einem besseren Verständnis der untersuchten Problematik bei und bietet praktische Lösungen, die in der Industrie angewendet werden können. Die entwickelten Methoden bilden eine solide Grundlage für zukünftige Forschung und praktische Anwendungen.

