% !TEX root = ../Bachelor-Thesis.tex

\chapter{Fazit und Ausblick}
\label{ch:abschluss}

\section{Zusammenfassung der Ergebnisse}

Diese Arbeit hat gezeigt, dass die in Kapitel \ref{ch:konzept} und \ref{ch:realisierung} entwickelte Lösung die identifizierten Probleme effektiv adressiert. Die durchgeführten Experimente validieren den gewählten Ansatz.

\subsection{Beantwortung der Forschungsfragen}

\textbf{Forschungsfrage 1:} Wie können existierende Methoden verbessert werden?

Die Arbeit zeigt, dass durch gezielte Optimierungen der Architektur und Algorithmen signifikante Verbesserungen erreicht werden können.

\textbf{Forschungsfrage 2:} Welche praktischen Auswirkungen hat der neue Ansatz?

In realen Szenarien zeigen sich Leistungssteigerungen von durchschnittlich 20\% sowie verbesserte Zuverlässigkeit.

\section{Beiträge dieser Arbeit}

Diese Arbeit leistet folgende Beiträge zur Forschung:

\begin{enumerate}
  \item Entwicklung eines neuen Lösungsansatzes basierend auf aktuellen Erkenntnissen
  \item Praktische Validierung durch umfangreiche Experimente
  \item Identifikation von Verbesserungspotentialen in bestehenden Methoden
  \item Publikation von Erkenntnissen für die wissenschaftliche Community
\end{enumerate}

\section{Limitierungen}

Trotz der positiven Ergebnisse gibt es folgende Limitierungen:

\begin{itemize}
  \item Die Experimente wurden in kontrollierter Umgebung durchgeführt
  \item Skalierungstests waren auf 10.000 Requests/min begrenzt
  \item Die Validierung konzentrierte sich auf spezifische Datensätze
\end{itemize}

\section{Zukünftige Arbeiten}

Auf Basis dieser Arbeit ergeben sich mehrere Richtungen für zukünftige Forschung:

\subsection{Kurzfristige Verbesserungen}

\begin{itemize}
  \item Optimierung der Speichernutzung für Echtzeit-Anwendungen
  \item Erweiterung der Testabdeckung auf weitere Datensätze
  \item Integration mit bestehenden Systemen
\end{itemize}

\subsection{Langfristige Perspektiven}

\begin{itemize}
  \item Erweiterung des Ansatzes auf verwandte Problemdomänen
  \item Untersuchung von Hybrid-Methoden
  \item Machine-Learning basierte Optimierungen
\end{itemize}

\section{Schlusswort}

Diese Arbeit trägt zu einem besseren Verständnis der untersuchten Problematik bei und bietet praktische Lösungen, die in der Industrie angewendet werden können. Die entwickelten Methoden bilden eine solide Grundlage für zukünftige Forschung und praktische Anwendungen.

