% !TEX root = ../Bachelor-Thesis.tex

\chapter{Konzept und Methodik}
\label{ch:konzept}

\section{Übersicht des Lösungsansatzes}

Auf Basis der Analyse in Kapitel \ref{ch:hintergrund} wird in diesem Kapitel ein neuer Lösungsansatz entwickelt. Der Ansatz adressiert die identifizierten Forschungslücken und nutzt die Stärken bestehender Methoden.

\section{Architektur und Design}

Das Konzept basiert auf folgenden Designprinzipien:

\begin{itemize}
  \item Modularität: Komponenten sind unabhängig und austauschbar
  \item Skalierbarkeit: Das System wächst mit den Anforderungen
  \item Wartbarkeit: Code ist verständlich und dokumentiert
  \item Robustheit: Fehlertoleranz und Zuverlässigkeit
\end{itemize}

\section{Mathematische Grundlagen}

Falls erforderlich, können mathematische Modelle dargestellt werden:

\[
  f(x) = \sum_{i=1}^{n} x_i \cdot w_i
\]

wobei $x_i$ die Eingabewerte und $w_i$ die Gewichtungen darstellen.

\section{Methodik}

Die Realisierung folgt einem systematischen Vorgehen:

\begin{enumerate}
  \item Anforderungsanalyse: Präzise Definition der Ziele
  \item Designphase: Architektur und Schnittstellen festlegen
  \item Implementierungsphase: Umsetzung des Designs
  \item Testphase: Validierung und Verifikation
  \item Optimierungsphase: Performance und Qualität verbessern
\end{enumerate}

\section{Abgrenzung zu alternativen Ansätzen}

Dieser Ansatz unterscheidet sich von den in Kapitel \ref{ch:hintergrund} beschriebenen Methoden durch:

\begin{itemize}
  \item Verbesserte Effizienz durch optimierte Algorithmen
  \item Bessere Skalierungseigenschaften
  \item Praktischere Anwendbarkeit
\end{itemize}
